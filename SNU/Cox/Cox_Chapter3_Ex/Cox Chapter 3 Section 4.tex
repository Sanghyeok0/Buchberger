\documentclass[11pt]{article}

\usepackage{amsmath, amssymb, amsthm}
\usepackage{enumitem}
\usepackage{mathtools}

\newtheorem{proposition}{Proposition}[section]

\theoremstyle{definition}
\newtheorem{definition}[proposition]{Definition}
\newtheorem{example}[proposition]{Example}

\begin{document}

\section*{Tangent Lines, Singular Points, and Envelopes}

%----------------------------------------
% Definition 1: intersection multiplicity along a line
%----------------------------------------
\begin{definition}\label{def:intersection-multiplicity}
Let $m$ be a positive integer.  
Assume that $(a,b) \in V(f)$ and let $L$ be the line through $(a,b)$.
We say that $L$ \emph{meets $V(f)$ with multiplicity $m$ at $(a,b)$}
if $L$ can be parametrized as
\begin{equation}\label{eq:line-param}
  \begin{aligned}
    x &= a + c t,\\
    y &= b + d t ,
  \end{aligned}
\end{equation}
and, for
\[
  g(t) := f(a + c t,\, b + d t),
\]
the number $t = 0$ is a root of multiplicity $m$ of the polynomial $g(t)$.
\end{definition}

%----------------------------------------
% Proposition 2
%----------------------------------------
\begin{proposition}\label{prop:tangent-multiplicity}
Let $f \in k[x,y]$, and let $(a,b) \in V(f)$.
\begin{enumerate}[label=(\roman*)]
  \item If $\nabla f(a,b) \ne (0,0)$, then there is a unique line through $(a,b)$
        that meets $V(f)$ with multiplicity at least $2$.
  \item If $\nabla f(a,b) = (0,0)$, then every line through $(a,b)$
        meets $V(f)$ with multiplicity at least $2$.
\end{enumerate}
\end{proposition}

%----------------------------------------
% Definition 3: tangent line, singular / nonsingular points
%----------------------------------------
\begin{definition}\label{def:tangent-singular}
Let $f \in k[x,y]$ and let $(a,b) \in V(f)$.
\begin{enumerate}[label=(\roman*)]
  \item Suppose $\nabla f(a,b) \ne (0,0)$.  
        The \emph{tangent line} of $V(f)$ at $(a,b)$ is the unique line through
        $(a,b)$ that meets $V(f)$ with multiplicity at least $2$.  
        In this situation we call $(a,b)$ a \emph{nonsingular point} of $V(f)$.
  \item If $\nabla f(a,b) = (0,0)$, then $(a,b)$ is called a
        \emph{singular point} of $V(f)$.
\end{enumerate}
\end{definition}

%----------------------------------------
% Definition 4: family of curves determined by F
%----------------------------------------
\begin{definition}\label{def:family-of-curves}
Let $F \in \mathbb{R}[x,y,t]$.  
For a fixed real number $t \in \mathbb{R}$, the variety in $\mathbb{R}^2$
defined by $F(x,y,t) = 0$ is denoted by $V(F_t)$.  
The \emph{family of curves determined by $F$} is the collection
of all varieties $V(F_t)$ as $t$ ranges over $\mathbb{R}$.
\end{definition}

\begin{example}\label{ex:translated-parabolas}
Consider
\[
  F(x,y,t) = (x - t)^2 - y + t .
\]
Writing this as $y - t = (x - t)^2$, we see that $V(F_t)$ is the parabola
obtained from $y = x^2$ by translating along the line $y = x$.
\end{example}

%----------------------------------------
% Definition 5: envelope of a family
%----------------------------------------
\begin{definition}\label{def:envelope}
Let $\{V(F_t)\}_{t \in \mathbb{R}}$ be the family of curves in $\mathbb{R}^2$
associated with $F \in \mathbb{R}[x,y,t]$ as in Definition~\ref{def:family-of-curves}.
The \emph{envelope} of this family is the set of all points $(x,y) \in \mathbb{R}^2$
for which there exists $t \in \mathbb{R}$ such that
\[
  F(x,y,t) = 0
  \qquad\text{and}\qquad
  \frac{\partial F}{\partial t}(x,y,t) = 0 .
\]
\end{definition}

\end{document}

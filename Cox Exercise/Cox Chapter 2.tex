\begin{exercise}[Ch.\ 2 \S4, Ex.\ 1]
Let $I \subseteq k[x_1,\dots,x_n]$ be an ideal with the following property:

For every $f=\sum_{\alpha} c_{\alpha}x^{\alpha}\in I$, every monomial $x^{\alpha}$
appearing in $f$ (i.e.\ with $c_{\alpha}\neq 0$) is also in $I$.

Show that $I$ is a monomial ideal.
\end{exercise}

\begin{proof}[Solution]
Let
\[
S := \{\, x^{\alpha} \mid x^{\alpha}\in I \,\}
\quad\text{and}\quad
J := \langle S\rangle
\]
be the ideal generated by all monomials that lie in $I$.
Since $S\subseteq I$ and $I$ is an ideal, we have $J\subseteq I$.

For the reverse inclusion, take $f=\sum_{\alpha}c_{\alpha}x^{\alpha}\in I$.
By the assumption, every monomial $x^{\alpha}$ with $c_{\alpha}\neq 0$ lies in $I$,
hence belongs to $S$; therefore each term $c_{\alpha}x^{\alpha}$ lies in $J$,
and so $f\in J$.
Thus $I\subseteq J$, and hence $I=J$.

Therefore $I$ is generated by monomials, i.e.\ $I$ is a monomial ideal.
\end{proof}
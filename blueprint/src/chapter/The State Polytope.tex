\chapter{The State Polytope}

\section{Basic Concepts of Polyhedral Geometry}

In the first half of this chapter we review some basic concepts from polyhedral
geometry. In the second half we introduce the state polytope of an ideal $I$. It
has the property that its vertices are in a natural bijection with the distinct initial
ideals $\mathrm{in}_{<}(I)$.

\begin{definition}[Polyhedron]\label{def:polyhedron}
    \lean{PolyhedralGeometry.HPolyhedron}
    \leanok
    A \emph{polyhedron} is a finite intersection of closed half-spaces in $\mathbb{R}^n$. Thus a polyhedron $P$ can be written as $P = \{ \mathbf{x} \in \mathbb{R}^n : A \cdot \mathbf{x} \le \mathbf{b} \}$, where $A$ is a matrix with $n$ columns.
\end{definition}

If $\mathbf{b} = 0$ then there exist vectors $\mathbf{u}_1, \dots, \mathbf{u}_m \in \mathbb{R}^n$ such that
\begin{equation}\label{eq:polyhedral_cone_V_rep}
P = \operatorname{pos}(\{\mathbf{u}_1, \dots, \mathbf{u}_m\}) := \{ \lambda_1 \mathbf{u}_1 + \dots + \lambda_m \mathbf{u}_m : \lambda_1, \dots, \lambda_m \in \mathbb{R}_+ \}.
\end{equation}

\begin{definition}[Polyhedral Cone]\label{def:polyhedral_cone}
    \lean{PolyhedralGeometry.PolyhedralCone}
    \leanok
    A polyhedron of the form \eqref{eq:polyhedral_cone_V_rep} is called a \emph{(polyhedral) cone}.
\end{definition}

\begin{lemma}[Characterization of Cone Membership]\label{lem:mem_polyhedral_cone}
    \lean{PolyhedralGeometry.mem_polyhedralCone_iff_exists_nonneg_coeffs}
    % The corresponding Lean lemma has `sorry`, so `\leanok` is omitted.
    A point $\mathbf{x}$ is in a polyhedral cone if and only if it is a non-negative linear combination of its generators, as defined in \eqref{eq:polyhedral_cone_V_rep}.
\end{lemma}

Here and throughout this book $\mathbb{R}_+$ denotes the non-negative reals. The \emph{polar} of a cone $C$ is defined as
\[
C^* = \{ \boldsymbol{\omega} \in \mathbb{R}^n : \boldsymbol{\omega} \cdot \mathbf{c} \le 0 \text{ for all } \mathbf{c} \in C \}.
\]

\begin{definition}[Polytope]\label{def:polytope}
    \lean{PolyhedralGeometry.Polytope}
    \leanok
    A polyhedron $Q$ which is bounded is called a \emph{polytope}. Every polytope $Q$ can be written as the convex hull of a finite set of points
    \begin{equation}\label{eq:polytope_convex_hull}
    Q = \operatorname{conv}(\{\mathbf{v}_1, \dots, \mathbf{v}_m\}) := \left\{ \sum_{i=1}^m \lambda_i \mathbf{v}_i : \lambda_1, \dots, \lambda_m \in \mathbb{R}_+, \sum_{i=1}^m \lambda_i = 1 \right\}.
    \end{equation}
\end{definition}

Here are two examples of 3-dimensional polytopes: The cube and the octahedron.
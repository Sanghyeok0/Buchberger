\chapter{Gr{\"o}bner Bases} 

\section{Polynomial Reductions}

\begin{theorem}
[Division Algorithm for Multivariate Polynomials]\label{thm:div_alg}
% [Becker] 198p Proposition 5.22 / [Cox] 64p Theorem 3
  \lean{MonomialOrder.div_set}
  \leanok
  Let $P$ be a subset of $K[X]$ and $f\in K[X]$.
  Then there exists a normal form $g\in K[X]$ of $f$ modulo $P$ and a family $\mathcal{F}=\{q_{p}\}_{p\in P}$ of elements of $K[X]$ with
  \[
    f=\sum_{p\in P}q_{p}\,p + g
    \quad\text{and}\quad
    \max\bigl\{\LT(q_{p}p)\mid p\in P,\;q_{p}p\neq 0\bigr\}\le \LT(f).
  \]
  If $P$ is finite, the ground field is computable, and the term order on $T$ is decidable, 
  then $g$ and $\{q_{p}\}_{p\in P}$ can be computed from $f$ and $P$.
\end{theorem}
\begin{proof}
  \leanok 
\end{proof}

\section{Gr{\"o}bner Bases-Existence and Uniqueness}

\begin{definition}[Initial Ideal]
  \label{def:initialIdeal}
  \lean{MonomialOrder.initialIdeal}
  \leanok 
  Let \(I\subseteq k[x_1,\dots,x_n]\) be an ideal other than \(\{0\}\), and fix a monomial ordering on \(k[x_1,\dots,x_n]\).  Then:
  \begin{enumerate}
    \item We denote by
    \[
      \mathrm{LT}(I)
      =
      \{\,c\,x^\alpha \mid \exists\,f\in I\setminus\{0\}\text{ with }\mathrm{LT}(f)=c\,x^\alpha\}.
    \]
    \item We denote by \(\langle \mathrm{LT}(I)\rangle\) the ideal generated by the elements of \(\mathrm{LT}(I)\).
  \end{enumerate}
\end{definition}

\begin{theorem} % [Cox] 77p Proposition 3
    \label{thm:initialIdeal_is_FG}
    \uses{def:initialIdeal, thm:Dickson_MvPolynomial}
    \lean{MonomialOrder.initialIdeal_is_monomial_ideal, MonomialOrder.initialIdeal_is_FG}
    \leanok 
    Let $I \subseteq k[x_1, \ldots , x_n]$ be an ideal different from $\{ 0 \}$.
    \begin{enumerate}
        \item $\langle\LT(I)\rangle$ is a monomial ideal.
        \item There are $g_1, \ldots , g_t \in I$ such that $\langle \LT(I)\rangle = \langle \LT(g_1), \ldots , \LT(g_t)\rangle$.
    \end{enumerate}
\end{theorem}
\begin{proof}
\leanok
\begin{enumerate}
    \item The leading monomials $\text{LM}(g)$ of elements $g \in I \setminus \{0\}$ generate the monomial ideal $\langle \text{LM}(g) \mid g \in I \setminus \{0\} \rangle$. 
      Since $\text{LM}(g)$ and $\text{LT}(g)$ differ by a nonzero constant, this ideal equals $\langle \text{LT}(g) \mid g \in I \setminus \{0\} \rangle = \langle \text{LT}(I) \rangle$. 
      Thus, $\langle \text{LT}(I) \rangle$ is a monomial ideal.
    \item Since $\langle \text{LT}(I) \rangle$ is generated by the monomials $\text{LM}(g)$ for $g \in I \setminus \{0\}$, Dickson's Lemma tells us that $\langle \text{LT}(I) \rangle = \langle \text{LM}(g_1), \dots, \text{LM}(g_t) \rangle$ for finitely many $g_1, \dots, g_t \in I$. 
      Since $\text{LM}(g_i)$ differs from $\text{LT}(g_i)$ by a nonzero constant, it follows that $\langle \text{LT}(I) \rangle = \langle \text{LT}(g_1), \dots, \text{LT}(g_t) \rangle$. 
      This completes the proof.
\end{enumerate}
\end{proof}

\begin{definition}\label{def:Groebner_basis}
    \uses{thm:initialIdeal_is_FG}
    \lean{MvPolynomial.IsGroebnerBasis}
    \leanok 
    Fix a monomial order $>$ on the polynomial ring $k[x_1, \dots, x_n]$. 
    A finite subset $G = \{g_1, \dots, g_t\}$ of an ideal $I \subseteq k[x_1, \dots, x_n]$ different from $\{0\}$ 
    is said to be a \textbf{Gr{\"o}bner basis} (or \textbf{standard basis}) for $I$ if the ideal generated by the
    leading terms of the elements in $G$ is equal to the ideal generated by the leading terms of all elements in $I$.
    That is,
    \[ \ideal{\LT(g_1), \dots, \LT(g_t)} = \ideal{\LT(I)}, \]
    where $\LT(I) = \{\LT(f) \mid f \in I \setminus \{0\}\}$.
    Using the convention that $\ideal{\emptyset} = \{0\}$, we define the empty set $\emptyset$
    to be the Gr{\"o}bner basis of the zero ideal $\{0\}$.
\end{definition}

\begin{proposition}\label{prop:GR_Unique_Remainder} % [Cox] 83p Proposition 1
    \uses{thm:div_alg, def:Groebner_basis, lem:mem_monomialIdeal_iff_divisible}
    \lean{MvPolynomial.normalForm_exists_unique}
    \leanok 
    Let $I \subseteq k[x_1,\dots,x_n]$ be an ideal and let $G = \{g_1,\dots,g_t\}$ be a Gr{\"o}bner basis for $I$.  
    Then given $f \in k[x_1,\dots,x_n]$ there is a unique $r \in k[x_1,\dots,x_n]$ with the following two properties:
    \begin{enumerate}
    \item No term of $r$ is divisible by any of 
      \(\mathrm{LT}(g_1),\dots,\mathrm{LT}(g_t)\).
    \item There is $g\in I$ such that $f = g + r$.
    \end{enumerate}
    In particular, $r$ is the remainder on division of $f$ by $G$ no matter how the elements of $G$ are listed when using the division algorithm.
\end{proposition}
\begin{proof}
  \leanok
  The division algorithm gives $f = q_1 g_1 + \cdots + q_t g_t + r$, where $r$ satisfies (i). We can also satisfy (ii) by setting $g = q_1 g_1 + \cdots + q_t g_t \in I$. 
  This proves the existence of $r$.
  To prove uniqueness, suppose $f = g+r = g'+r'$ satisfy (i) and (ii). 
  Then $r-r' = g' - g \in I$, so that if $r \ne r'$, then $\text{LT}(r-r') \in \langle \text{LT}(I) \rangle = \langle \text{LT}(g_1), \dots, \text{LT}(g_t) \rangle$. 
  By Lemma~\ref{lem:mem_monomialIdeal_iff_divisible}, it follows that $\text{LT}(r-r')$ is divisible by some $\text{LT}(g_i)$. This is impossible since no term of $r, r'$ is divisible by one of $\text{LT}(g_1), \dots, \text{LT}(g_t)$. Thus $r-r'$ must be zero, and uniqueness is proved.
  The final part of the proposition follows from the uniqueness of $r$.
\end{proof}

\begin{corollary}[Ideal Membership Problem]
    \label{cor:GB_membership_test} % [Cox] 84p Corollary 2
    \uses{prop:GR_Unique_Remainder}
    \lean{MvPolynomial.mem_Ideal_iff_GB_normalForm_eq_zero}
    \leanok
    Let $G = \{g_1, \dots , g_t\}$ be a Gr{\"o}bner basis for an ideal $I \subseteq k[x_1, \dots , x_n]$
    (with respect to a given monomial order $>$) and let $f \in k[x_1, \dots , x_n]$.
    Then $f \in I$ if and only if the remainder on division of $f$ by
    $G$ is zero.
    \[ f \in I \iff \rem(f, G) = 0. \]
\end{corollary}
\begin{proof}
  \leanok
  If the remainder is zero, then we have already observed that $f \in I$. 
  Conversely, given $f \in I$, then $f = f + 0$ satisfies the two conditions of Proposition~\ref{prop:GR_Unique_Remainder}. 
  It follows that $0$ is the remainder of $f$ on division by $G$.
\end{proof}


\begin{definition}\label{def:Remainder} % [Cox] 84p Definition 3
    \uses{thm:div_alg}
    \lean{MvPolynomial.normalForm}
    \leanok 
    We will write \(\overline{f}^F\) for the remainder(normalform) on division of \(f\) by the ordered \(s\)-tuple
    \[
    F \;=\;(f_1,\dots,f_s).
    \]
    If \(F\) is a Gr{\"o}bner basis for the ideal \(\langle f_1,\dots,f_s\rangle\), then by Proposition 1
    we may regard \(F\) as a set (without any particular order).
\end{definition}

\begin{definition}\label{def:S-polynomial} % [Cox] 84p Definition 4
    \lean{MvPolynomial.S_polynomial}
    \leanok 
    Let \(f,g\in k[x_1,\dots,x_n]\) be nonzero polynomials.
    \begin{enumerate}
      \item If \(\operatorname{multideg}(f)=\alpha\) and \(\operatorname{multideg}(g)=\beta\), 
        then let 
        \[
          \gamma = (\gamma_1,\dots,\gamma_n),
          \quad
          \gamma_i = \max(\alpha_i,\beta_i)
          \quad\text{for each }i.
        \]
        We call \(x^\gamma\) the \emph{least common multiple} of \(\operatorname{LM}(f)\) and \(\operatorname{LM}(g)\),
        written
        \[
          x^\gamma \;=\;\operatorname{lcm}\bigl(\operatorname{LM}(f),\,\operatorname{LM}(g)\bigr).
        \]
      \item The \(S\)-polynomial of \(f\) and \(g\) is the combination
        \[
          S(f,g)
          \;=\;
          \frac{x^\gamma}{\mathrm{LT}(f)}\,f
          \;-\;
          \frac{x^\gamma}{\mathrm{LT}(g)}\,g.
        \]
    \end{enumerate}
\end{definition}

\begin{lemma}\label{lem:Spolynomial_syzygy_of_degree_cancellation} % [Cox] 85p Lemma 5
    \uses{thm:div_alg, def:S-polynomial}
    \lean{MvPolynomial.Spolynomial_syzygy_of_degree_cancellation}
    \leanok 
    Suppose we have a sum $\sum_{i=1}^s p_i,$ where $\operatorname{multideg}(p_i)=\delta\in\mathbb Z_{\ge0}^n$ for all $i$.
    If $\operatorname{multideg}\Bigl(\sum_{i=1}^s p_i\Bigr)<\delta,$ then $\sum_{i=1}^s p_i$ is a linear combination, with coefficients in~$k$, of the $S$-polynomials $S(p_j,p_l)\quad\text{for }1\le j,\,l\le s$.
    Furthermore, each $S(p_j,p_l)$ has multidegree $<\delta$.
\end{lemma}
\begin{proof}
  \leanok
  Let $d_i = \text{LC}(p_i)$, so that $d_i x^\delta$ is the leading term of $p_i$. Since the sum $\sum_{i=1}^s p_i$ has strictly smaller multidegree, it follows easily that $\sum_{i=1}^s d_i = 0$.

  Next observe that since $p_i$ and $p_j$ have the same leading monomial, their S-polynomial reduces to
  \begin{equation} \tag{1}
  S(p_i, p_j) = \frac{1}{d_i}p_i - \frac{1}{d_j}p_j.
  \end{equation}

  It follows that
  \begin{equation} \tag{2}
  \begin{aligned}
  \sum_{i=1}^{s-1} d_i S(p_i, p_s) &= d_1\left(\frac{1}{d_1}p_1 - \frac{1}{d_s}p_s\right) + d_2\left(\frac{1}{d_2}p_2 - \frac{1}{d_s}p_s\right) + \cdots \\
  &= p_1 + p_2 + \cdots + p_{s-1} - \frac{1}{d_s}(d_1 + \cdots + d_{s-1})p_s.
  \end{aligned}
  \end{equation}
  However, $\sum_{i=1}^s d_i = 0$ implies $d_1 + \cdots + d_{s-1} = -d_s$, so that (2) reduces to
  \[
  \sum_{i=1}^{s-1} d_i S(p_i, p_s) = p_1 + \cdots + p_{s-1} + p_s.
  \]
  Thus, $\sum_{i=1}^s p_i$ is a sum of S-polynomials of the desired form, and equation (1) makes it easy to see that $S(p_i, p_j)$ has multidegree $<\delta$. The lemma is proved.
\end{proof}

\begin{lemma}\label{lem:S-polynomials_and_Monomial_Multiplication} % [Cox] 89p Exercise 8
  \uses{def:S-polynomial}
  \lean{MvPolynomial.Spolynomial_of_monomial_mul_eq_monomial_mul_Spolynomial}
  \leanok 
  Suppose that $p_i = c_i x^{\alpha^{(i)}} g_i$ and $p_j = c_j x^{\alpha^{(j)}} g_j$ have the same multidegree $\delta$. Then
  \[
  S(p_i, p_j) = x^{\delta - \gamma_{ij}} S(g_i, g_j),
  \]
  where $x^{\gamma_{ij}} = \operatorname{lcm}(\operatorname{LM}(g_i), \operatorname{LM}(g_j))$.
\end{lemma}
\begin{proof}
  \leanok
  By hypothesis, $\operatorname{multideg}(p_i) = \operatorname{multideg}(p_j) = \delta$, which implies $\delta = \alpha^{(i)} + \operatorname{multideg}(g_i)$ and $\delta = \alpha^{(j)} + \operatorname{multideg}(g_j)$.
  The leading terms are $\operatorname{LT}(p_i) = c_i \operatorname{LC}(g_i) x^\delta$ and $\operatorname{LT}(p_j) = c_j \operatorname{LC}(g_j) x^\delta$.
  Let $\gamma_{ij}$ be the exponent of $\operatorname{lcm}(\operatorname{LM}(g_i), \operatorname{LM}(g_j))$.

  On the one hand, the left-hand side simplifies to:
  \begin{align*}
  S(p_i, p_j) &= \frac{x^\delta}{\operatorname{LT}(p_i)} p_i - \frac{x^\delta}{\operatorname{LT}(p_j)} p_j \\
  &= \frac{x^\delta}{c_i \operatorname{LC}(g_i) x^\delta} (c_i x^{\alpha^{(i)}} g_i) - \frac{x^\delta}{c_j \operatorname{LC}(g_j) x^\delta} (c_j x^{\alpha^{(j)}} g_j) \\
  &= \frac{1}{\operatorname{LC}(g_i)} x^{\alpha^{(i)}} g_i - \frac{1}{\operatorname{LC}(g_j)} x^{\alpha^{(j)}} g_j.
  \end{align*}

  On the other hand, the right-hand side expands to:
  \begin{align*}
  x^{\delta - \gamma_{ij}} S(g_i, g_j)
  &= x^{\delta - \gamma_{ij}} \left( \frac{x^{\gamma_{ij}}}{\operatorname{LT}(g_i)} g_i - \frac{x^{\gamma_{ij}}}{\operatorname{LT}(g_j)} g_j \right) \\
  &= \frac{x^{\delta}}{\operatorname{LC}(g_i)x^{\operatorname{multideg}(g_i)}} g_i - \frac{x^{\delta}}{\operatorname{LC}(g_j)x^{\operatorname{multideg}(g_j)}} g_j \\
  &= \frac{1}{\operatorname{LC}(g_i)} x^{\delta - \operatorname{multideg}(g_i)} g_i - \frac{1}{\operatorname{LC}(g_j)} x^{\delta - \operatorname{multideg}(g_j)} g_j \\
  &= \frac{1}{\operatorname{LC}(g_i)} x^{\alpha^{(i)}} g_i - \frac{1}{\operatorname{LC}(g_j)} x^{\alpha^{(j)}} g_j.
  \end{align*}
  The two sides are equal, which completes the proof.
\end{proof}

\begin{theorem}[Buchberger's Criterion] % [Cox] 86p Buchberger’s Criterion % [Becker-Weispfenning1993] 213p Corollary 5.52
    \label{thm:Buchbergers_Criterion}
    \uses{lem:degree_sum_le_syn, cor:GB_membership_test, lem:Spolynomial_syzygy_of_degree_cancellation, lem:S-polynomials_and_Monomial_Multiplication, lem:sum_ite_not_mem}
    \lean{MvPolynomial.Buchberger_criterion}
    \leanok 
    Let $I$ be a polynomial ideal in $k[x_1, \dots, x_n]$. 
    Then a basis $G = \{g_1, \dots , g_t\}$ of $I$ is a Gr{\"o}bner basis for $I$ (with respect to a given monomial order $>$)
    if and only if for all pairs $i \neq j$, the remainder on division of the $S$-polynomial $S(g_i, g_j)$ by $G$ (listed in some order) is zero.
    % \[ \forall i \neq j, \quad S(g_i, g_j) \xrightarrow{G} 0 \]
    % or equivalently,
    \[ \forall i \neq j, \quad \rem(S(g_i, g_j), G) = 0. \]
\end{theorem}

\begin{proof}
  \leanok
  $\Rightarrow$: If $G$ is a Gr\"obner basis, then since $S(g_i, g_j) \in I$, the remainder on division by $G$ is zero by Corollary~\ref{cor:GB_membership_test}.

  $\Leftarrow$: Let $f \in I$ be nonzero. We will show that $\text{LT}(f) \in \langle\text{LT}(g_1), \dots, \text{LT}(g_t)\rangle$ as follows. Write
  \[
  f = \sum_{i=1}^t h_i g_i, \quad h_i \in k[x_1, \dots, x_n].
  \]
  From Lemma~\ref{lem:degree_sum_le}, it follows that
  \begin{equation} \tag{3}
  \text{multideg}(f) \le \max(\text{multideg}(h_i g_i) \mid h_i g_i \ne 0).
  \end{equation}
  The strategy of the proof is to pick the most efficient representation of $f$, meaning that among all expressions $f = \sum_{i=1}^t h_i g_i$, we pick one for which
  \[
  \delta = \max(\text{multideg}(h_i g_i) \mid h_i g_i \ne 0)
  \]
  is minimal. The minimal $\delta$ exists by the well-ordering property of our monomial ordering. By (3), it follows that $\text{multideg}(f) \le \delta$.

  If equality occurs, then $\text{multideg}(f) = \text{multideg}(h_i g_i)$ for some $i$. 
  This easily implies that $\text{LT}(f)$ is divisible by $\text{LT}(g_i)$. Then $\text{LT}(f) \in \langle\text{LT}(g_1), \dots, \text{LT}(g_t)\rangle$, which is what we want to prove.

  It remains to consider the case when the minimal $\delta$ satisfies $\text{multideg}(f) < \delta$. 
  We will use $S(g_i, g_j) \overline{\hspace{1.5em}}^G 0$ for $i \ne j$ to find a new expression for $f$ that decreases $\delta$. 
  This will contradict the minimality of $\delta$ and complete the proof.

  Given an expression $f = \sum_{i=1}^t h_i g_i$ with minimal $\delta$, we begin by isolating the part of the sum where multidegree $\delta$ occurs:
  \begin{align} \tag{4}
  f &= \sum_{\text{multideg}(h_i g_i) = \delta} h_i g_i + \sum_{\text{multideg}(h_i g_i) < \delta} h_i g_i \\
  &= \sum_{\text{multideg}(h_i g_i) = \delta} \text{LT}(h_i) g_i + \sum_{\text{multideg}(h_i g_i) = \delta} (h_i - \text{LT}(h_i)) g_i + \sum_{\text{multideg}(h_i g_i) < \delta} h_i g_i. \nonumber
  \end{align}
  The monomials appearing in the second and third sums on the second line all have multidegree $<\delta$. 
  Then $\text{multideg}(f) < \delta$ means that the first sum on the second line also has multidegree $<\delta$.

  The key to decreasing $\delta$ is to rewrite the first sum in two stages: use Lemma~\ref{lem:Spolynomial_syzygy_of_degree_cancellation} to rewrite the first sum in terms of S-polynomials, and then use $S(g_i, g_j) \overline{\hspace{1.5em}}^G 0$ to rewrite the S-polynomials without cancellation.

  To express the first sum on the second line of (4) using S-polynomials, note that
  \begin{equation} \tag{5}
  \sum_{\text{multideg}(h_i g_i) = \delta} \text{LT}(h_i) g_i
  \end{equation}
  satisfies the hypothesis of Lemma~\ref{lem:Spolynomial_syzygy_of_degree_cancellation} since each $p_i = \text{LT}(h_i) g_i$ has multidegree $\delta$ and the sum has multidegree $<\delta$.
  Hence, the first sum is a linear combination with coefficients in $k$ of the S-polynomials $S(p_i, p_j)$.
  In Exercise~\ref{lem:S-polynomials_and_Monomial_Multiplication}, you will verify that
  \[
  S(p_i, p_j) = x^{\delta - \gamma_{ij}} S(g_i, g_j),
  \]
  where $x^{\gamma_{ij}} = \text{lcm}(\text{LM}(g_i), \text{LM}(g_j))$. It follows that the first sum (5) is a linear combination of $S(g_i, g_j)$ for certain pairs $(i, j)$.

  Consider one of these S-polynomials $S(g_i, g_j)$. Since $S(g_i, g_j) \overline{\hspace{1.5em}}^G 0$, the division algorithm (Theorem 3 of §3) gives an expression
  \begin{equation} \tag{6}
  S(g_i, g_j) = \sum_{l=1}^t A_l g_l,
  \end{equation}
  where $A_l \in k[x_1, \dots, x_n]$ and
  \begin{equation} \tag{7}
  \text{multideg}(A_l g_l) \le \text{multideg}(S(g_i, g_j))
  \end{equation}
  when $A_l g_l \ne 0$. Now multiply each side of (6) by $x^{\delta - \gamma_{ij}}$ to obtain
  \begin{equation} \tag{8}
  x^{\delta - \gamma_{ij}} S(g_i, g_j) = \sum_{l=1}^t B_l g_l,
  \end{equation}
  where $B_l = x^{\delta - \gamma_{ij}} A_l$. Then (7) implies that when $B_l g_l \ne 0$, we have
  \begin{equation} \tag{9}
  \text{multideg}(B_l g_l) \le \text{multideg}(x^{\delta - \gamma_{ij}} S(g_i, g_j)) < \delta
  \end{equation}
  since $\text{LT}(S(g_i, g_j)) < \text{lcm}(\text{LM}(g_i), \text{LM}(g_j)) = x^{\gamma_{ij}}$ by Exercise 7.

  It follows that the first sum (5) is a linear combination of certain $x^{\delta - \gamma_{ij}} S(g_i, g_j)$, each of which satisfies (8) and (9). Hence we can write the first sum as
  \begin{equation} \tag{10}
  \sum_{\text{multideg}(h_i g_i) = \delta} \text{LT}(h_i) g_i = \sum_{l=1}^t \tilde{B}_l g_l
  \end{equation}
  with the property that when $\tilde{B}_l g_l \ne 0$, we have
  \begin{equation} \tag{11}
  \text{multideg}(\tilde{B}_l g_l) < \delta.
  \end{equation}
  Substituting (10) into the second line of (4) gives an expression for $f$ as a polynomial combination of the $g_i$'s where \textit{all} terms have multidegree $<\delta$. This contradicts the minimality of $\delta$ and completes the proof.
\end{proof}

\begin{definition}\label{def:reduces_to_zero} % [Cox] 104p
    \lean{MvPolynomial.reduces_to_zero}
    Fix a monomial order and let $G = \left\{g_1, \ldots , g_t\right\} \subseteq k[x_1, \ldots , x_n]$.
    Given $f \in k[x_1, \ldots , x_n]$, we say that $f$ \textbf{reduces to zero modulo} $G$, written $f \xrightarrow{G} 0$,
    if $f$ has a \textbf{standard representation}
    \[ f = A_1g_1 + \cdots A_tg_t,\ A_i \in k[x_1, \ldots , x_n],\]
    which means that whenever $A_ig_i \neq 0$, we have
    \[\operatorname{deg}(f) \geq \operatorname{deg}(A_ig_i).\]
\end{definition}

\begin{theorem}[Buchberger's Algorithm]
  \label{thm:Buchbergers_Algorithm}
  \uses{thm:Buchbergers_Criterion}
  \lean{MvPolynomial.Buchberger_Alg}
  \leanok 
  Let $I = \langle f_1, \ldots, f_s \rangle \ne \{0\}$ be a polynomial ideal. Then a Gr{\"o}bner basis for $I$ can be constructed in a finite number of steps by the following algorithm:
  \normalfont % Switch from the theorem's default italic font to normal font
  \begin{itemize}
      \item[] \textbf{Input :} $F = (f_1, \ldots, f_s)$
      \item[] \textbf{Output :} A Gr{\"o}bner basis $G = (g_1, \ldots, g_t)$ for $I$, with $F \subseteq G$
      \vspace{1ex}
      \item[] $G := F$
      \item[] \textbf{REPEAT}
      \begin{itemize}
          \item[] $G' := G$
          \item[] \textbf{FOR} each pair $\{p, q\}$, $p \ne q$ in $G'$ \textbf{DO}
          \begin{itemize}
              \item[] $r := \overline{S(p, q)}^{G'}$
              \item[] \textbf{IF} $r \ne 0$ \textbf{THEN} $G := G \cup \{r\}$
          \end{itemize}
      \end{itemize}
      \item[] \textbf{UNTIL} $G = G'$
      \item[] \textbf{RETURN} $G$
  \end{itemize}
\end{theorem}
\begin{proof}
  \leanok
  We begin with some frequently used notation. If $G = \{ g_1, \ldots, g_t \}$, 
  then $\langle G \rangle$ and $\langle \operatorname{LT}(G) \rangle$ will denote the following ideals:
  \[
  \langle G \rangle = \langle g_1, \ldots, g_t \rangle, \qquad
  \langle \operatorname{LT}(G) \rangle = \langle \operatorname{LT}(g_1), \ldots, \operatorname{LT}(g_t) \rangle.
  \]

  Turning to the proof of the theorem, we first show that $G \subseteq I$ holds at every stage of
  the algorithm. This is true initially, and whenever we enlarge $G$, we do so by adding the remainder
  $r = \overline{S(p,q)}^{G'}$ for $p,q \in G' \subseteq G$. Thus, if $G \subseteq I$, then $p,q \in I$ and, 
  hence, $S(p,q) \in I$, and since we are dividing by $G' \subseteq I$, we get $G \cup \{r\} \subseteq I$. 
  We also note that $G$ contains the given basis $F$ of $I$, so that $G$ is actually a basis of $I$.

  The algorithm terminates when $G = G'$, which means that 
  $r = \overline{S(p,q)}^{G'} = 0$ for all $p,q \in G$. Hence $G$ is a Gr\"obner basis of 
  $\langle G \rangle = I$ by Theorem~6 of \S6.

  It remains to prove that the algorithm terminates. We need to consider what happens after each
  pass through the main loop. The set $G$ consists of $G'$ (the old $G$) together with the nonzero
  remainders of $S$-polynomials of elements of $G'$. Then
  \begin{equation}
  \langle \operatorname{LT}(G') \rangle \subseteq \langle \operatorname{LT}(G) \rangle.
  \end{equation}
  Since $G' \subseteq G$. Furthermore, if $G' \ne G$, we claim that 
  $\langle \operatorname{LT}(G') \rangle$ is strictly smaller than $\langle \operatorname{LT}(G) \rangle$. 
  To see this, suppose that a nonzero remainder $r$ of an $S$-polynomial has been adjoined to $G$. 
  Since $r$ is a remainder on division by $G'$, $\operatorname{LT}(r)$ is not divisible by the leading
  terms of elements of $G'$, and thus $\operatorname{LT}(r) \notin \langle \operatorname{LT}(G') \rangle$ 
  by Lemma~2 of \S4. Yet $\operatorname{LT}(r) \in \langle \operatorname{LT}(G) \rangle$, which proves our claim.

  By (1), the ideals $\langle \operatorname{LT}(G') \rangle$ from successive iterations of the loop form an
  ascending chain of ideals in $k[x_1, \ldots, x_n]$. Thus, the ACC (Theorem~7 of \S5) implies that after
  a finite number of iterations the chain will stabilize, so that 
  $\langle \operatorname{LT}(G') \rangle = \langle \operatorname{LT}(G) \rangle$ must happen eventually. 
  By the previous paragraph, this implies that $G' = G$, so that the algorithm must terminate after a 
  finite number of steps.
\end{proof}

\begin{lemma}\label{lem:grobner_basis_remove_redundant} % [Cox] 92p
    \lean{MvPolynomial.grobner_basis_remove_redundant}
    Let $G$ be a Gr{\"o}bner basis of $I \subseteq k[x_1, \ldots, x_n]$.  
    Let $p \in G$ be a polynomial such that $\LT(p) \in \langle \LT(G \setminus \{p\}) \rangle.$  
    Then $G \setminus \{p\}$ is also a Gr{\"o}bner basis for $I$.
\end{lemma}
\begin{proof}
  \leanok
  We know that $\langle\operatorname{LT}(G)\rangle = \langle\operatorname{LT}(I)\rangle$. 
  If $\operatorname{LT}(p) \in \langle\operatorname{LT}(G \setminus \{p\})\rangle$, then we have $\langle\operatorname{LT}(G \setminus \{p\})\rangle = \langle\operatorname{LT}(G)\rangle$.
  By definition, it follows that $G \setminus \{p\}$ is also a Gr\"{o}bner basis for $I$.
\end{proof}
    
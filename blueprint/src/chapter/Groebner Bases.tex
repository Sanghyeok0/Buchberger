\chapter{Gr{\"o}bner Bases} 

\section{Polynomial Reductions}

\begin{theorem}
[Division Algorithm for Multivariate Polynomials]\label{thm:div_alg}
% [Becker] 198p Proposition 5.22 / [Cox] 64p Theorem 3
  \lean{MonomialOrder.div_set}
  \leanok
  Let $P$ be a subset of $K[X]$ and $f\in K[X]$.
  Then there exists a normal form $g\in K[X]$ of $f$ modulo $P$ and a family $\mathcal{F}=\{q_{p}\}_{p\in P}$ of elements of $K[X]$ with
  \[
    f=\sum_{p\in P}q_{p}\,p + g
    \quad\text{and}\quad
    \max\bigl\{\LT(q_{p}p)\mid p\in P,\;q_{p}p\neq 0\bigr\}\le \LT(f).
  \]
  If $P$ is finite, the ground field is computable, and the term order on $T$ is decidable, 
  then $g$ and $\{q_{p}\}_{p\in P}$ can be computed from $f$ and $P$.
\end{theorem}
\begin{proof}
  \leanok %확인 필요
\end{proof}

\section{Gr{\"o}bner Bases-Existence and Uniqueness}

\begin{definition}[Initial Ideal]
  \label{def:initialIdeal}
  \lean{MonomialOrder.initialIdeal}
  \leanok %확인 필요
  Let \(I\subseteq k[x_1,\dots,x_n]\) be an ideal other than \(\{0\}\), and fix a monomial ordering on \(k[x_1,\dots,x_n]\).  Then:
  \begin{enumerate}[(i)]
    \item We denote by
    \[
      \mathrm{LT}(I)
      =
      \{\,c\,x^\alpha \mid \exists\,f\in I\setminus\{0\}\text{ with }\mathrm{LT}(f)=c\,x^\alpha\}.
    \]
    \item We denote by \(\langle \mathrm{LT}(I)\rangle\) the ideal generated by the elements of \(\mathrm{LT}(I)\).
  \end{enumerate}
\end{definition}

\begin{theorem} % [Cox] 77p Proposition 3
    \label{thm:initialIdeal_is_FG}
    \uses{def:initialIdeal, thm:Dickson_MvPolynomial}
    \lean{MonomialOrder.initialIdeal_is_monomial_ideal, MonomialOrder.initialIdeal_is_FG}
    \leanok %확인 필요
    Let $I \subseteq k[x_1, \ldots , x_n]$ be an ideal different from $\{ 0 \}$.
    \begin{enumerate}
        \item $\langle\LT(I)\rangle$ is a monomial ideal.
        \item There are $g_1, \ldots , g_t \in I$ such that $\langle \LT(I)\rangle = \langle \LT(g_1), \ldots , \LT(g_t)\rangle$.
    \end{enumerate}
\end{theorem}
\begin{proof}
  \leanok %확인 필요
\end{proof}

\begin{definition}\label{def:Groebner_basis}
    \uses{thm:initialIdeal_is_FG}
    \lean{MvPolynomial.IsGroebnerBasis}
    \leanok %확인 필요
    Fix a monomial order $>$ on the polynomial ring $k[x_1, \dots, x_n]$. 
    A finite subset $G = \{g_1, \dots, g_t\}$ of an ideal $I \subseteq k[x_1, \dots, x_n]$ different from $\{0\}$ 
    is said to be a \textbf{Gr{\"o}bner basis} (or \textbf{standard basis}) for $I$ if the ideal generated by the
    leading terms of the elements in $G$ is equal to the ideal generated by the leading terms of all elements in $I$.
    That is,
    \[ \ideal{\LT(g_1), \dots, \LT(g_t)} = \ideal{\LT(I)}, \]
    where $\LT(I) = \{\LT(f) \mid f \in I \setminus \{0\}\}$.
    Using the convention that $\ideal{\emptyset} = \{0\}$, we define the empty set $\emptyset$
    to be the Gr{\"o}bner basis of the zero ideal $\{0\}$.
\end{definition}

\begin{proposition}\label{prop:GR_Unique_Remainder} % [Cox] 83p Proposition 1
    \uses{thm:div_alg, def:Groebner_basis, lem:mem_monomialIdeal_iff_divisible}
    \lean{MvPolynomial.normalForm_exists_unique}
    \leanok %확인 필요
    Let $I \subseteq k[x_1,\dots,x_n]$ be an ideal and let $G = \{g_1,\dots,g_t\}$ be a Gr{\"o}bner basis for $I$.  
    Then given $f \in k[x_1,\dots,x_n]$ there is a unique $r \in k[x_1,\dots,x_n]$ with the following two properties:
    \begin{enumerate}
    \item No term of $r$ is divisible by any of 
      \(\mathrm{LT}(g_1),\dots,\mathrm{LT}(g_t)\).
    \item There is $g\in I$ such that $f = g + r$.
    \end{enumerate}
    In particular, $r$ is the remainder on division of $f$ by $G$ no matter how the elements of $G$ are listed when using the division algorithm.
\end{proposition}
\begin{proof}
  \leanok %확인 필요
\end{proof}

\begin{corollary}[Ideal Membership Problem]
    \label{cor:GB_membership_test} % [Cox] 84p Corollary 2
    \uses{prop:GR_Unique_Remainder}
    \lean{MvPolynomial.mem_Ideal_iff_GB_normalForm_eq_zero}
    \leanok
    Let $G = \{g_1, \dots , g_t\}$ be a Gr{\"o}bner basis for an ideal $I \subseteq k[x_1, \dots , x_n]$
    (with respect to a given monomial order $>$) and let $f \in k[x_1, \dots , x_n]$.
    Then $f \in I$ if and only if the remainder on division of $f$ by
    $G$ is zero.
    \[ f \in I \iff \rem(f, G) = 0. \]
\end{corollary}
\begin{proof}
  \leanok %확인 필요
\end{proof}

\begin{definition}\label{def:Remainder} % [Cox] 84p Definition 3
    \uses{thm:div_alg}
    \lean{MvPolynomial.normalForm}
    \leanok %확인 필요
    We will write \(\overline{f}^F\) for the remainder(normalform) on division of \(f\) by the ordered \(s\)-tuple
    \[
    F \;=\;(f_1,\dots,f_s).
    \]
    If \(F\) is a Gr{\"o}bner basis for the ideal \(\langle f_1,\dots,f_s\rangle\), then by Proposition 1
    we may regard \(F\) as a set (without any particular order).
\end{definition}

\begin{definition}\label{def:S-polynomial} % [Cox] 84p Definition 4
    \lean{MvPolynomial.S_polynomial}
    \leanok %확인 필요
    Let \(f,g\in k[x_1,\dots,x_n]\) be nonzero polynomials.
    \begin{enumerate}
      \item If \(\operatorname{multideg}(f)=\alpha\) and \(\operatorname{multideg}(g)=\beta\), 
        then let 
        \[
          \gamma = (\gamma_1,\dots,\gamma_n),
          \quad
          \gamma_i = \max(\alpha_i,\beta_i)
          \quad\text{for each }i.
        \]
        We call \(x^\gamma\) the \emph{least common multiple} of \(\operatorname{LM}(f)\) and \(\operatorname{LM}(g)\),
        written
        \[
          x^\gamma \;=\;\operatorname{lcm}\bigl(\operatorname{LM}(f),\,\operatorname{LM}(g)\bigr).
        \]
      \item The \(S\)-polynomial of \(f\) and \(g\) is the combination
        \[
          S(f,g)
          \;=\;
          \frac{x^\gamma}{\mathrm{LT}(f)}\,f
          \;-\;
          \frac{x^\gamma}{\mathrm{LT}(g)}\,g.
        \]
    \end{enumerate}
\end{definition}

\begin{lemma}\label{lem:Spolynomial_syzygy_of_degree_cancellation} % [Cox] 85p Lemma 5
    \uses{thm:div_alg, def:S-polynomial}
    \lean{MvPolynomial.Spolynomial_syzygy_of_degree_cancellation}
    \leanok %확인 필요
    Suppose we have a sum $\sum_{i=1}^s p_i,$ where $\operatorname{multideg}(p_i)=\delta\in\mathbb Z_{\ge0}^n$ for all $i$.
    If $\operatorname{multideg}\Bigl(\sum_{i=1}^s p_i\Bigr)<\delta,$ then $\sum_{i=1}^s p_i$ is a linear combination, with coefficients in~$k$, of the $S$-polynomials $S(p_j,p_l)\quad\text{for }1\le j,\,l\le s$.
    Furthermore, each $S(p_j,p_l)$ has multidegree $<\delta$.
\end{lemma}
\begin{proof}
  \leanok %확인 필요
\end{proof}

\begin{lemma}\label{lem:S-polynomials_and_Monomial_Multiplication} % [Cox] 89p Exercise 8
  \uses{def:S-polynomial}
  \lean{MvPolynomial.Spolynomial_of_monomial_mul_eq_monomial_mul_Spolynomial}
  \leanok %확인 필요
  Suppose that $c_i x^{\alpha^{(i)}} g_i$ and $c_j x^{\alpha^{(j)}} g_j$ have multidegree $\delta$. Then
  \[
  S(x^{\alpha^{(i)}} g_i, x^{\alpha^{(j)}} g_j) = x^{\delta - \gamma_{ij}} S(g_i, g_j),
  \]
  where $x^{\gamma_{ij}} = \operatorname{lcm}(\operatorname{LM}(g_i), \operatorname{LM}(g_j))$.
\end{lemma}

\begin{theorem}[Buchberger's Criterion] % [Cox] 86p Buchberger’s Criterion % [Becker-Weispfenning1993] 213p Corollary 5.52
    \label{thm:Buchberger’s_Criterion} 
    \uses{lem:degree_sum_le_syn, cor:GB_membership_test, lem:Spolynomial_syzygy_of_degree_cancellation, lem:S-polynomials_and_Monomial_Multiplication, lem:sum_ite_not_mem}
    \lean{MvPolynomial.Buchberger_criterion}
    \leanok %확인 필요
    Let $I$ be a polynomial ideal in $k[x_1, \dots, x_n]$. 
    Then a basis $G = \{g_1, \dots , g_t\}$ of $I$ is a Gr{\"o}bner basis for $I$ (with respect to a given monomial order $>$)
    if and only if for all pairs $i \neq j$, the remainder on division of the $S$-polynomial $S(g_i, g_j)$ by $G$ (listed in some order) is zero.
    % \[ \forall i \neq j, \quad S(g_i, g_j) \xrightarrow{G} 0 \]
    % or equivalently,
    \[ \forall i \neq j, \quad \rem(S(g_i, g_j), G) = 0. \]
\end{theorem}

\begin{definition}\label{def:reduces_to_zero} % [Cox] 104p
    \lean{MvPolynomial.reduces_to_zero}
    Fix a monomial order and let $G = \left\{g_1, \ldots , g_t\right\} \subseteq k[x_1, \ldots , x_n]$.
    Given $f \in k[x_1, \ldots , x_n]$, we say that $f$ \textbf{reduces to zero modulo} $G$, written $f \xrightarrow{G} 0$,
    if $f$ has a \textbf{standard representation}
    \[ f = A_1g_1 + \cdots A_tg_t,\ A_i \in k[x_1, \ldots , x_n],\]
    which means that whenever $A_ig_i \neq 0$, we have
    \[\operatorname{deg}(f) \geq \operatorname{deg}(A_ig_i).\]
\end{definition}

\begin{theorem}[Buchberger's Algorithm]
  \label{thm:Buchberger's_Algorithm}
  \uses{thm:Buchberger’s_Criterion}
  \lean{MvPolynomial.Buchberger_Algorithm}
  Let $I = \langle f_1, \ldots, f_s \rangle \ne \{0\}$ be a polynomial ideal. Then a Gr{\"o}bner basis for $I$ can be constructed in a finite number of steps by the following algorithm:
  \normalfont % Switch from the theorem's default italic font to normal font
  \begin{itemize}
      \item[] \textbf{Input :} $F = (f_1, \ldots, f_s)$
      \item[] \textbf{Output :} A Gr{\"o}bner basis $G = (g_1, \ldots, g_t)$ for $I$, with $F \subseteq G$
      \vspace{1ex}
      \item[] $G := F$
      \item[] \textbf{REPEAT}
      \begin{itemize}
          \item[] $G' := G$
          \item[] \textbf{FOR} each pair $\{p, q\}$, $p \ne q$ in $G'$ \textbf{DO}
          \begin{itemize}
              \item[] $r := \overline{S(p, q)}^{G'}$
              \item[] \textbf{IF} $r \ne 0$ \textbf{THEN} $G := G \cup \{r\}$
          \end{itemize}
      \end{itemize}
      \item[] \textbf{UNTIL} $G = G'$
      \item[] \textbf{RETURN} $G$
  \end{itemize}
\end{theorem}

\begin{lemma}\label{lem:grobner_basis_remove_redundant} % [Cox] 92p
    \lean{MvPolynomial.grobner_basis_remove_redundant}
    Let $G$ be a Gr{\"o}bner basis of $I \subseteq k[x_1, \ldots, x_n]$.  
    Let $p \in G$ be a polynomial such that $\LT(p) \in \langle \LT(G \setminus \{p\}) \rangle.$  
    Then $G \setminus \{p\}$ is also a Gr{\"o}bner basis for $I$.
\end{lemma}
    
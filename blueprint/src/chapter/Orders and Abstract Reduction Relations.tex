\chapter{Orders and Abstract Reduction Relations} 

\section{Monomial Ideals and Dickson’s Lemma}

\begin{definition}\label{def:Orders}
    Let $r$ be a relation on $M$.
    Then $r$ is called
    \begin{enumerate}
        \item \textbf{reflexive} if $\Delta(M) \subseteq r$,
        \item \textbf{symmetric} if $r \subseteq r^{-1}$,
        \item \textbf{transitive} if $r \circ r \subseteq r$,
        \item \textbf{antisymmetric} if $r \cap r^{-1} \subseteq \Delta(M)$,
        \item \textbf{connex} if $r \cup r^{-1} = M \times M$,
        \item \textbf{irreflexive} if $\Delta(M) \cap r = \emptyset$,
        \item \textbf{strictly antisymmetric} if $r \cap r^{-1} = \emptyset$,
        \item an \textbf{equivalence relation} on $M$ if $r$ is reflexive, symmetric, and transitive,
        \item a \textbf{quasi-order} (\textbf{preorder}) on $M$ if $r$ is reflexive and transitive,
        \item a \textbf{partial order} on $M$ if $r$ is reflexive, transitive and antisymmetric,
        \item a \textbf{(linear) order} on $M$ if $r$ is a connex partial order on $M$, and
        \item a \textbf{linear quasi-order} on $M$ if $r$ is a connex quasi-order on $M$.
    \end{enumerate}
    
    \[
    \begin{tikzcd}[row sep=huge, column sep=small]
    & \textbf{quasi-orders} = \textbf{preorders} \arrow[dr] \arrow[dl]& \\
    \textbf{linear quasi-orders} \arrow[dr] & & \textbf{partial orders} \arrow[dl] \\
    & \textbf{(linear) orders} = \textbf{total orders} &
    \end{tikzcd}
    \]
\end{definition}
    
\begin{definition}\label{def:well-founded}
    Let $r$ be a relation on $M$ with strict part $r_s$, and let $N \subseteq M$.
    \begin{enumerate}
        \item Then an element $a$ of $N$ is called \textit{$r$-minimal} (\textit{$r$-maximal}) in $N$ if there is no $b \in N$ with $b \, r_s \, a$ (with $a \, r_s \, b$).
        For $N = M$ the reference to $N$ is omitted.
    
        \item A \textit{strictly descending} (\textit{strictly ascending}) $r$-chain in $M$ is an infinite sequence $\{a_n\}_{n \in \mathbb{N}}$ of elements of $M$ such that $a_{n+1} \, r_s \, a_n$ (such that $a_n \, r_s \, a_{n+1}$) for all $n \in \mathbb{N}$.
    
        \item The relation $r$ is called \textbf{well-founded} (\textbf{noetherian}) if every non-empty subset $N$ of $M$ has an $r$-minimal (an $r$-maximal) element.
        $r$ is a \textbf{well-order} on $M$ if $r$ is a well-founded linear order on $M$.
    \end{enumerate}
\end{definition}
    
\begin{definition}\label{def:wqo}
    Let $\preceq$ be a quasi-order on $M$ and let $N \subseteq M$. 
    Then a subset $B$ of $N$ is called a \textbf{Dickson basis}, or simply \textbf{basis} of $N$ w.r.t.\ $\preceq$, if for every $a \in N$ there exists some $b \in B$ with $b \preceq a$.
    \begin{enumerate}
        \item We say that $\preceq$ has the \textbf{Dickson property}, or is a \textbf{well-quasi-order}(wqo), if every subset $N$ of $M$ has a finite basis w.r.t.\ $\preceq$.
        \item A \textbf{well partial order}, or a wpo, is a wqo that is a proper ordering relation, i.e., it is antisymmetric.
    \end{enumerate}
\end{definition}
    
\begin{proposition}\label{prop:wqoEquivalent}
    Let $\preceq$ be a quasi-order on $M$ with associated equivalence relation $\sim$. 
    Then the following are equivalent:
    \begin{enumerate}[label=(\roman*)]
        \item $\preceq$ is a well-quasi-order.
        \item Whenever $\{a_n\}_{n \in \mathbb{N}}$ is a sequence of elements of $M$, then there exist $i < j$ with $a_i \preceq a_j$.
        \item \textit{For every nonempty subset $N$ of $M$, the number of min-classes in $N$ is finite and non-zero.}
    \end{enumerate}
\end{proposition}
    
\begin{corollary}\label{cor:MinimalFiniteBasis}
    Let $\preceq$ be a well-partial-order on $M$.
    Then every non-empty subset $N$ of $M$ has a \textup{\textbf{unique minimal finite basis}} $B$, i.e., a finite basis $B$ such that $B \subseteq C$ for all other bases $C$ of $N$. 
    $B$ consists of all minimal elements of $N$.
\end{corollary}
    
\begin{corollary}\label{cor:wqo_is_wellFounded}
    Every well-quasi-order is well-founded.
\end{corollary}
    
\begin{proposition}\label{prop:wqoAscendingSubsequence}
    Let $\preceq$ be a well- quasi-order on $M$, and let $\{a_n\}_{n \in \mathbb{N}}$ be a sequence of elements of $M$.
    Then there exists a strictly ascending sequence $\{n_i\}_{i \in \mathbb{N}}$ of natural numbers such that $a_{n_i} \preceq a_{n_j}$ for all $i < j$.
\end{proposition}

\begin{lemma}\label{lem:monomial_ideal}
    \lean{MvPolynomial.initialIdeal_is_monomial_ideal}
    Let $I = \langle x^\alpha \mid \alpha \in A \rangle$ be a monomial ideal.
    Then a monomial $x^\beta$ lies in $I$ if and only if $x^\beta$ is divisible by $x^\alpha$ for some $\alpha \in A$.
\end{lemma}

\begin{theorem}\label{thm:Dickson} % [Becker-Weispfenning1993] 163p
    \lean{MvPolynomial.Dickson_lemma}
    Let $(\mathbb{N}^n, \le')$ be the direct product of $n$ copies of the natural numbers $(\mathbb{N}, \le)$ with their natural ordering. 
    Then $(\mathbb{N}^n, \le')$ is a Dickson partially ordered set. 
    More explicitly, every subset $S \subseteq \mathbb{N}^n$ has a finite subset $B$ such that for every $(m_1, \dots, m_n) \in S$, there exists $(k_1, \dots, k_n) \in B$ with
    \[
    k_i \le m_i \quad \text{for } 1 \le i \le n.
    \]
\end{theorem}

\begin{theorem}[Dickson's Lemma (MvPolynomial)]\label{thm:Dickson'}
    \uses{thm:Dickson}
    \lean{MvPolynomial.Dickson_lemma'}
    Every subset $S\subseteq \mathbb{N}^n$ has only finitely many minimal elements with respect to the componentwise partial order. 
    In particular, every monomial ideal in $k[x_1,\dots,x_n]$ is finitely generated.
\end{theorem}

\begin{theorem}
    test
\end{theorem}
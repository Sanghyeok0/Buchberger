\chapter{Orders and Abstract Reduction Relations} 

\section{Monomial Ideals and Dickson’s Lemma}

\begin{definition}\label{def:Orders}
    \lean{Preorder, LinearOrder}
    \mathlibok
    Let $r$ be a relation on $M$.
    Then $r$ is called
    \begin{enumerate}
        \item \textbf{reflexive} if $\Delta(M) \subseteq r$,
        \item \textbf{symmetric} if $r \subseteq r^{-1}$,
        \item \textbf{transitive} if $r \circ r \subseteq r$,
        \item \textbf{antisymmetric} if $r \cap r^{-1} \subseteq \Delta(M)$,
        \item \textbf{connex} if $r \cup r^{-1} = M \times M$,
        \item \textbf{irreflexive} if $\Delta(M) \cap r = \emptyset$,
        \item \textbf{strictly antisymmetric} if $r \cap r^{-1} = \emptyset$,
        \item an \textbf{equivalence relation} on $M$ if $r$ is reflexive, symmetric, and transitive,
        \item a \textbf{quasi-order} (\textbf{preorder}) on $M$ if $r$ is reflexive and transitive,
        \item a \textbf{partial order} on $M$ if $r$ is reflexive, transitive and antisymmetric,
        \item a \textbf{(linear) order} on $M$ if $r$ is a connex partial order on $M$, and
        \item a \textbf{linear quasi-order} on $M$ if $r$ is a connex quasi-order on $M$.
    \end{enumerate}
    
    \[
    \begin{tikzcd}[row sep=huge, column sep=small]
    & \textbf{quasi-orders} = \textbf{preorders} \arrow[dr] \arrow[dl]& \\
    \textbf{linear quasi-orders} \arrow[dr] & & \textbf{partial orders} \arrow[dl] \\
    & \textbf{(linear) orders} = \textbf{total orders} &
    \end{tikzcd}
    \]
\end{definition}
    
\begin{definition}\label{def:well-founded}
    \lean{WellFounded}
    \mathlibok
    \uses{def:Orders}
    Let $r$ be a relation on $M$ with strict part $r_s$, and let $N \subseteq M$.
    \begin{enumerate}
        \item Then an element $a$ of $N$ is called \textit{$r$-minimal} (\textit{$r$-maximal}) in $N$ if there is no $b \in N$ with $b \, r_s \, a$ (with $a \, r_s \, b$).
        For $N = M$ the reference to $N$ is omitted.
    
        \item A \textit{strictly descending} (\textit{strictly ascending}) $r$-chain in $M$ is an infinite sequence $\{a_n\}_{n \in \mathbb{N}}$ of elements of $M$ such that $a_{n+1} \, r_s \, a_n$ (such that $a_n \, r_s \, a_{n+1}$) for all $n \in \mathbb{N}$.
    
        \item The relation $r$ is called \textbf{well-founded} (\textbf{noetherian}) if every non-empty subset $N$ of $M$ has an $r$-minimal (an $r$-maximal) element.
        $r$ is a \textbf{well-order} on $M$ if $r$ is a well-founded linear order on $M$.
    \end{enumerate}
\end{definition}

\begin{definition}[The ``Antisymmetrization'' of $M$]\label{def:eq_rel}
    \uses{def:Orders}
    \lean{Antisymmetrization}
    \mathlibok %확인 필요
    Let $(M,\le)$ be a preordered set.  Define an equivalence relation
    \[
      \sim\;\colon\;M\times M\;\to\;\mathrm{Prop},
      \qquad
      a\sim b \;\iff\; \bigl(a\le b \wedge b\le a\bigr).
    \]
    We write $[a]$ for the equivalence class of $a$, and denote the quotient by
    \[
      \mathrm{Quot}(M,\sim)\;=\;\{\, [a]\mid a\in M\}.
    \]
\end{definition}
    
\begin{definition}[Minimal elements and min–classes]\label{def:min–classes}
    \uses{def:eq_rel}
    \lean{minClasses}
    \leanok %확인 필요
    Let $N\subseteq M$.  An element $b\in N$ is called \emph{minimal in $N$} if
    \[
      \forall\,y\in N,\;y\le b \;\Longrightarrow\; b\le y.
    \]
    We denote by
    \[
      \operatorname{Minimal}(N)
      \;=\;
      \{\,b\in N \mid \forall\,y\in N,\;y\le b\to b\le y\}
    \]
    the set of all minimal elements of $N$.  The \emph{min–classes} of $N$ are then
    \[
      \operatorname{minClasses}(N)
      \;=\;
      \bigl\{\, [b]\in \mathrm{Quot}(M,\sim)\;\bigm|\;b\in \operatorname{Minimal}(N)\bigr\}.
    \]
\end{definition}
    
\begin{definition}\label{def:wqo}
    \lean{HasDicksonProperty}
    \leanok %확인 필요
    \uses{def:Orders}
    Let $\preceq$ be a quasi-order on $M$ and let $N \subseteq M$. 
    Then a subset $B$ of $N$ is called a \textbf{Dickson basis}, or simply \textbf{basis} of $N$ w.r.t.\ $\preceq$, if for every $a \in N$ there exists some $b \in B$ with $b \preceq a$.
    \begin{enumerate}
        \item We say that $\preceq$ has the \textbf{Dickson property}, or is a \textbf{well-quasi-order}(wqo), if every subset $N$ of $M$ has a finite basis w.r.t.\ $\preceq$.
        \item A \textbf{well partial order}, or a wpo, is a wqo that is a proper ordering relation, i.e., it is antisymmetric.
    \end{enumerate}
\end{definition}
    
\begin{proposition}\label{prop:wqoEquivalent} % [Becker-Weispfenning1993] 160p Proposition 4.42
    \uses{def:wqo, def:min–classes}
    \lean{HasDicksonProperty.to_wellQuasiOrderedLE, WellQuasiOrderedLE.minClasses_finite_and_nonempty, finite_min_classes_implies_hasDicksonProperty}
    \leanok %확인 필요
    Let $\preceq$ be a quasi-order on $M$ with associated equivalence relation $\sim$. 
    Then the following are equivalent:
    \begin{enumerate}
        \item $\preceq$ is a well-quasi-order.
        \item Whenever $\{a_n\}_{n \in \mathbb{N}}$ is a sequence of elements of $M$, then there exist $i < j$ with $a_i \preceq a_j$.
        \item \textit{For every nonempty subset $N$ of $M$, the number of min-classes in $N$ is finite and non-zero.}
    \end{enumerate}
\end{proposition}
\begin{proof}
  \leanok %확인 필요
\end{proof}
    
\begin{corollary}\label{cor:MinimalFiniteBasis} % [Becker-Weispfenning1993] 161p Corollary 4.43
    \uses{prop:wqoEquivalent}
    Let $\preceq$ be a well-partial-order on $M$.
    Then every non-empty subset $N$ of $M$ has a \textup{\textbf{unique minimal finite basis}} $B$, i.e., a finite basis $B$ such that $B \subseteq C$ for all other bases $C$ of $N$. 
    $B$ consists of all minimal elements of $N$.
\end{corollary}
    
\begin{corollary}\label{cor:wqo_is_wellFounded} % [Becker-Weispfenning1993] 161p Corollary 4.44
    \uses{prop:wqoEquivalent}
    Every well-quasi-order is well-founded.
\end{corollary}
    
\begin{proposition}\label{prop:wqoAscendingSubsequence} % [Becker-Weispfenning1993] 160p Proposition 4.45
    \lean{WellQuasiOrdered.exists_monotone_subseq}
    \leanok
    \uses{prop:wqoEquivalent}
    Let $\preceq$ be a well- quasi-order on $M$, and let $\{a_n\}_{n \in \mathbb{N}}$ be a sequence of elements of $M$.
    Then there exists a strictly ascending sequence $\{n_i\}_{i \in \mathbb{N}}$ of natural numbers such that $a_{n_i} \preceq a_{n_j}$ for all $i < j$.
\end{proposition}
\begin{proof}
  We define the sequence $\{n_i\}_{i \in \mathbb{N}}$ recursively, and by simultaneous induction on $i$ we verify the following properties:
  \begin{enumerate}
      \item $a_{n_i} \preceq a_{n_{i+1}}$ for all $i \in \mathbb{N}$, and
      \item for all $i \in \mathbb{N}$, the set $\{n \in \mathbb{N} \mid a_{n_i} \preceq a_n\}$ is infinite.
  \end{enumerate}
  For $i=0$, let $\{b_1, \dots, b_k\}$ be a finite basis of the set $\{a_n \mid n \in \mathbb{N}\}$, and for each $j$ with $1 \le j \le k$, set
  \[
  B_j = \{n \in \mathbb{N} \mid b_j \preceq a_n\}.
  \]
  Then $\bigcup_{j=1}^k B_j = \mathbb{N}$ by the choice of $B$. 
  Since the union of finitely many finite sets is finite, we can find a $B_j$ which is infinite. 
  Moreover, $b_j = a_m$ for some $m \in \mathbb{N}$, and we set $n_0 = m$. 
  For $i+1$, we consider the set
  \[
  U_i = \{a_n \mid a_{n_i} \preceq a_n, n_i < n\}.
  \]
  By condition (ii) for $i$, the set $\{n \in \mathbb{N} \mid a_n \in U_i\}$ is infinite. 
  Choosing some finite basis of $U_i$, we can, as before, find an element $a_m$ in this basis such that $a_m \preceq a_n$ for infinitely many different $n \in \mathbb{N}$, and we take $n_{i+1} = m$. 
  Conditions (i) and (ii) obviously continue to hold. 
  It now follows easily from condition (i) and the transitivity of $\preceq$ that $\{n_i\}_{i \in \mathbb{N}}$ has the desired property.
\end{proof}

\begin{lemma}\label{lem:degree_sum_le} % [Cox] 60p Lemma 8
  \lean{MonomialOrder.degree_sum_le}
  \leanok %확인 필요
  Let $f, g \in k[x_1, \dots, x_n]$ be nonzero polynomials. Then:
  \begin{enumerate}
      \item $\operatorname{multideg}(fg) = \operatorname{multideg}(f) + \operatorname{multideg}(g)$.
      \item If $f + g \neq 0$, then $\operatorname{multideg}(f + g) \le \max(\operatorname{multideg}(f), \operatorname{multideg}(g))$. 
        If, in \textit{addition}, $\operatorname{multideg}(f) \neq \operatorname{multideg}(g)$, then \textit{equality occurs}.
  \end{enumerate}
\end{lemma}

\begin{lemma}\label{lem:degree_sum_le_syn}
  \lean{MonomialOrder.degree_sum_le_syn}
  \leanok %확인 필요
  Let $\iota$ be an index set and $s \subset \iota$ a finite subset. For each $i \in s$, let $h_i \in k[x_1,\dots,x_n]$. 
  Then the following inequality holds:
  \[
  \operatorname{multideg}\left(\sum_{i \in s} h_i\right) \le \max_{i \in s} \left\{ \operatorname{multideg}(h_i) \right\}
  \]
  where the $\max$ is taken with respect to the monomial order.
\end{lemma}

\begin{lemma}\label{lem:mem_monomialIdeal_iff_divisible} % [Cox] 70p Lemma 2
    \lean{MonomialOrder.mem_monomialIdeal_iff_divisible}
    \leanok %확인 필요
    Let $I = \langle x^\alpha \mid \alpha \in A \rangle$ be a monomial ideal.
    Then a monomial $x^\beta$ lies in $I$ if and only if $x^\beta$ is divisible by $x^\alpha$ for some $\alpha \in A$.
\end{lemma}
\begin{proof}
  \leanok
  If $x^\beta$ is a multiple of $x^\alpha$ for some $\alpha \in A$, then $x^\beta \in I$ by the definition of ideal. 
  Conversely, if $x^\beta \in I$, then $x^\beta = \sum_{i=1}^s h_i x^{\alpha(i)}$, where $h_i \in k[x_1, \dots, x_n]$ and $\alpha(i) \in A$. 
  If we expand each $h_i$ as a sum of terms, we obtain
  \[
  x^\beta = \sum_{i=1}^s h_i x^{\alpha(i)} = \sum_{i=1}^s \left(\sum_j c_{i,j} x^{\beta(i,j)}\right) x^{\alpha(i)} = \sum_{i,j} c_{i,j} x^{\beta(i,j)} x^{\alpha(i)}.
  \]
\end{proof}

\begin{theorem}\label{thm:Dickson} % [Becker-Weispfenning1993] 163p
    \uses{prop:wqoEquivalent, prop:wqoAscendingSubsequence}
    \lean{MonomialOrder.Dickson_lemma}
    \leanok %확인 필요
    Let $(\mathbb{N}^n, \le')$ be the direct product of $n$ copies of the natural numbers $(\mathbb{N}, \le)$ with their natural ordering. 
    Then $(\mathbb{N}^n, \le')$ is a Dickson partially ordered set. 
    More explicitly, every subset $S \subseteq \mathbb{N}^n$ has a finite subset $B$ such that for every $(m_1, \dots, m_n) \in S$, there exists $(k_1, \dots, k_n) \in B$ with
    \[
    k_i \le m_i \quad \text{for } 1 \le i \le n.
    \]
\end{theorem}
\begin{proof}
  \leanok 
\end{proof}

\begin{theorem}[Dickson's Lemma (MvPolynomial)]
    \label{thm:Dickson_MvPolynomial}
    \uses{thm:Dickson}
    \lean{MonomialOrder.Dickson_lemma_MV}
    \leanok %확인 필요
    % Every subset $S \subseteq \mathbb{N}^n$ has only finitely many minimal elements with respect to the componentwise partial order. 
    % In particular, every monomial ideal in $k[x_1,\dots, x_n]$ is finitely generated.
    Let $I = \langle x^{\alpha} | \alpha \in A \rangle \subseteq k[x_1, \ldots, x_n]$ be a monomial ideal.
    Then $I$ can be written in the form $I = \langle x^{\alpha(1)}, \ldots , {\alpha(s)} \rangle$, where
    $\alpha(1), \ldots, \alpha(s) \in A$.
    In particular, $I$ has a finite basis.
\end{theorem}
\begin{proof}
  \leanok 
\end{proof}